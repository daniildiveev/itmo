\documentclass{beamer}
\usepackage[english, russian]{babel}

\begin{document}
\hspace*{\fill} \Large{\color{blue}П\color{black}реобразование из  СС-\emph{N} в СС-$N^k$ и обратно

\vspace{10}
\small{
\color{teal}Из  СС-\emph{N} в СС-$N^k$

\scalebox{0.9}{%
\vbox{%
\color{black}\begin{itemize}
    \item дополнить число, записанное в СС с основанием \emph{N}, незначащими нулями так, чтобы количество цифр было равно \emph{k};
    \item разбить полученное число на группы по \emph{k} цифр, начиная с нуля;
    \item заменить каждую такую группу эквивалентным числом, записанным в СС в основанием $N^k$
\end{itemize}}}

\color{black}
\hspace{20}\textbf{Задача}: $1020101_{(3)} = ?_{(27)}$

\hspace{20}\textbf{Решение}: $1020101_{(3)} = 001 020 101_{(3)} = 16А_{(27)}$
}
\vspace{10}

\color{teal} Из CC-$N^k$ в СС-\emph{N}
\scalebox{0.9}{%
\vbox{%
\color{black}\begin{itemize}
    \item заменить каждую цифру числа, записанного в СС с основанием $N^k$, эквивалентным набором из \emph{k} цифр СС с основанием N
\end{itemize}}}

\color{black}
\hspace{20}\textbf{Задача}: $2345_{(125)} = ?_{(5)}$

\hspace{20}\textbf{Решение}: $2345_{(125)} = 002 003 004 010_{(5)} = 2003004010_{(5)}$
\end{document}